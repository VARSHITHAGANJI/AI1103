\documentclass[journal,12pt,twocolumn]{IEEEtran}

\usepackage{setspace}
\usepackage{gensymb}
\singlespacing
\usepackage[cmex10]{amsmath}

\usepackage{amsthm}

\usepackage{mathrsfs}
\usepackage{txfonts}
\usepackage{stfloats}
\usepackage{bm}
\usepackage{cite}
\usepackage{cases}
\usepackage{subfig}
\usepackage{tikz}
\usetikzlibrary{automata, positioning}
\usepackage{longtable}
\usepackage{multirow}

\usepackage{enumitem}
\usepackage{mathtools}
\usepackage{steinmetz}
\usepackage{tikz}
\usepackage{circuitikz}
\usepackage{verbatim}
\usepackage{tfrupee}
\usepackage[breaklinks=true]{hyperref}
\usepackage{graphicx}
\usepackage{tkz-euclide}


\usetikzlibrary{calc,math}
\usepackage{listings}
    \usepackage{color}                                            %%
    \usepackage{array}                                            %%
    \usepackage{longtable}                                        %%
    \usepackage{calc}                                             %%
    \usepackage{multirow}                                         %%
    \usepackage{hhline}                                           %%
    \usepackage{ifthen}                                           %%
    \usepackage{lscape}     
\usepackage{multicol}
\usepackage{chngcntr}

\DeclareMathOperator*{\Res}{Res}

\renewcommand\thesection{\arabic{section}}
\renewcommand\thesubsection{\thesection.\arabic{subsection}}
\renewcommand\thesubsubsection{\thesubsection.\arabic{subsubsection}}

\renewcommand\thesectiondis{\arabic{section}}
\renewcommand\thesubsectiondis{\thesectiondis.\arabic{subsection}}
\renewcommand\thesubsubsectiondis{\thesubsectiondis.\arabic{subsubsection}}


\hyphenation{op-tical net-works semi-conduc-tor}
\def\inputGnumericTable{}                                 %%

\lstset{
%language=C,
frame=single, 
breaklines=true,
columns=fullflexible
}

\begin{document}


\newtheorem{theorem}{Theorem}[section]
\newtheorem{problem}{Problem}
\newtheorem{proposition}{Proposition}[section]
\newtheorem{lemma}{Lemma}[section]
\newtheorem{corollary}[theorem]{Corollary}
\newtheorem{example}{Example}[section]
\newtheorem{definition}[problem]{Definition}

\newcommand{\BEQA}{\begin{eqnarray}}
\newcommand{\EEQA}{\end{eqnarray}}
\newcommand{\define}{\stackrel{\triangle}{=}}
\bibliographystyle{IEEEtran}
\raggedbottom
\setlength{\parindent}{0pt}
\providecommand{\mbf}{\mathbf}
\providecommand{\pr}[1]{\ensuremath{\Pr\left(#1\right)}}
\providecommand{\qfunc}[1]{\ensuremath{Q\left(#1\right)}}
\providecommand{\sbrak}[1]{\ensuremath{{}\left[#1\right]}}
\providecommand{\lsbrak}[1]{\ensuremath{{}\left[#1\right.}}
\providecommand{\rsbrak}[1]{\ensuremath{{}\left.#1\right]}}
\providecommand{\brak}[1]{\ensuremath{\left(#1\right)}}
\providecommand{\lbrak}[1]{\ensuremath{\left(#1\right.}}
\providecommand{\rbrak}[1]{\ensuremath{\left.#1\right)}}
\providecommand{\cbrak}[1]{\ensuremath{\left\{#1\right\}}}
\providecommand{\lcbrak}[1]{\ensuremath{\left\{#1\right.}}
\providecommand{\rcbrak}[1]{\ensuremath{\left.#1\right\}}}
\theoremstyle{remark}
\newtheorem{rem}{Remark}
\newcommand{\sgn}{\mathop{\mathrm{sgn}}}
\providecommand{\abs}[1]{$\left\vert#1\right\vert$}
\providecommand{\res}[1]{\Res\displaylimits_{#1}} 
\providecommand{\norm}[1]{$\left\lVert#1\right\rVert$}
%\providecommand{\norm}[1]{\lVert#1\rVert}
\providecommand{\mtx}[1]{\mathbf{#1}}
\providecommand{\mean}[1]{E$\left[ #1 \right]$}
\providecommand{\fourier}{\overset{\mathcal{F}}{ \rightleftharpoons}}
%\providecommand{\hilbert}{\overset{\mathcal{H}}{ \rightleftharpoons}}
\providecommand{\system}{\overset{\mathcal{H}}{ \longleftrightarrow}}
	%\newcommand{\solution}[2]{\textbf{Solution:}{#1}}
\newcommand{\solution}{\noindent \textbf{Solution: }}
\newcommand{\cosec}{\,\text{cosec}\,}
\providecommand{\dec}[2]{\ensuremath{\overset{#1}{\underset{#2}{\gtrless}}}}
\newcommand{\myvec}[1]{\ensuremath{\begin{pmatrix}#1\end{pmatrix}}}
\newcommand{\mydet}[1]{\ensuremath{\begin{vmatrix}#1\end{vmatrix}}}
\numberwithin{equation}{subsection}
\makeatletter
\@addtoreset{figure}{problem}
\makeatother
\let\StandardTheFigure\thefigure
\let\vec\mathbf
\renewcommand{\thefigure}{\theproblem}
\def\putbox#1#2#3{\makebox[0in][l]{\makebox[#1][l]{}\raisebox{\baselineskip}[0in][0in]{\raisebox{#2}[0in][0in]{#3}}}}  \def\rightbox#1{\makebox[0in][r]{#1}}
     \def\centbox#1{\makebox[0in]{#1}}
     \def\topbox#1{\raisebox{-\baselineskip}[0in][0in]{#1}}
     \def\midbox#1{\raisebox{-0.5\baselineskip}[0in][0in]{#1}}
\vspace{3cm}
\title{\textbf{PROBABILITY AND RANDOM VARIABLES \\ Assignment 3}}
\author{GANJI VARSHITHA - AI20BTECH11009}
\maketitle
\newpage
\bigskip
\renewcommand{\thefigure}{\theenumi}
\renewcommand{\thetable}{\theenumi}
Download latex-tikz codes from 
%
\begin{lstlisting}
https://github.com/VARSHITHAGANJI/AI1103_Probability_Assignment/blob/main/Assignment3.tex
\end{lstlisting}
\section*{Problem}
\textbf{GATE 2016 (MA) Question 48}
\\
Let $X_{1},X_{2},X_{3},\cdots$ be a sequence of i.i.d uniform \brak{0,1} random variables. Then the value of 
\begin{align} \lim_{n\to\infty} \pr{-\ln{\brak{1-X_{1}}}-\cdots-\ln{\brak{1-X_{n}}} > n} 
\end{align} is equal to 
\section*{Solution}
\begin{align}
f_{X_{i}}\brak{x} ={}&\begin{cases}
1 & 0<x<1\\
0 & \text{otherwise}
\end{cases}
\end{align}
Let $Y_{1},Y_{2},\cdots,$ be another sequence of random variables where $Y_{i} = -\ln{\brak{1-X_{i}}},  i=1,2,3,\cdots$
\\ 
\begin{align}
f_{Y_{i}}\brak{x}={}& \frac{f_{X_{i}}\brak{x}}{\frac{dY_{i}}{dX_{i}}}\\
f_{Y_{i}}\brak{x}={}&\begin{cases}
e^{-x} & x>0\\
0 & \text{otherwise}
\end{cases}
\end{align}
From the above probability function, we have all $Y_{i}'$s to be exponential random variables.\\
\begin{align}
Y_{i}\sim \text{Exp}\brak{1}\\
\Rightarrow \mu = 1, \sigma^2 = 1
\end{align}
The required probability is 
\begin{align}
\lim_{n\to\infty}\pr{\sum_{i=1}^{n}Y_{i}>n}\\
=\lim_{n\to\infty}\pr{\overline{Y_{n}}>1}
\end{align}
Consider \begin{align}
Z=\lim_{n\to\infty}\sqrt{n}\brak{\frac{\overline{Y_{n}}-\mu}{\sigma}}
\end{align}
\\ Since $\overline{Y_{n}}>1$, we have $Z>0$.\\
By central limit theorem, we have Z to be a standard normal distribution.
\begin{align}
Z \sim {\mathcal {N}}\brak{0 ,1}\\
\lim_{n\to\infty}\pr{\overline{Y_{n}}>1}={}&\pr{Z>0}\\
={}&\frac{1}{2}
\end{align}

\begin{align}
\therefore \lim_{n\to\infty} \pr{-\ln{\brak{1-X_{1}}}-\cdots-\ln{\brak{1-X_{n}}} > n}=0.5
\end{align}


\end{document}